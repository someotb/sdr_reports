% Начало преамбулы
\usepackage[english, russian]{babel}
\usepackage{fontspec} 
\setmainfont{Times New Roman} 
\setmonofont{Consolas} 

\usepackage[left=20mm, right=10mm, top=20mm, bottom=20mm, headheight=1cm, headsep=0.5cm, footskip=1cm]{geometry}
\pagestyle{plain}

\usepackage{indentfirst}
\setlength{\parindent}{1.25cm}
\setlength{\parskip}{0pt}

% Оформление глав и разделов
\setsecnumdepth{subsection}
\renewcommand*{\chapterheadstart}{}
\renewcommand*{\printchaptername}{}
\renewcommand*{\chapnumfont}{\normalfont\bfseries}
\renewcommand*{\afterchapternum}{\hspace{1em}}
\renewcommand*{\printchaptertitle}{\normalfont\bfseries\centering\MakeUppercase}
\setbeforesecskip{20pt}
\setaftersecskip{20pt}
\setsecheadstyle{\raggedright\normalfont\bfseries}
\setbeforesubsecskip{20pt}
\setaftersubsecskip{20pt}
\setsubsecheadstyle{\raggedright\normalfont\bfseries}

% Оглавление
\addto\captionsrussian{\renewcommand\contentsname{Содержание}}
\setrmarg{2.55em plus1fil}
\renewcommand{\aftertoctitle}{\afterchaptertitle \vspace{-\cftbeforechapterskip}}
\renewcommand*{\cftchapternumwidth}{1.5em}
\renewcommand*{\cftchapterfont}{\normalfont\MakeUppercase}
\renewcommand*{\cftchapterpagefont}{\normalfont}
\renewcommand*{\cftchapterdotsep}{\cftdotsep}
\renewcommand*{\cftdotsep}{1}
\renewcommand*{\cftchapterleader}{\cftdotfill{\cftchapterdotsep}}
\maxtocdepth{subsection}

% Графика
\usepackage{graphicx}
\graphicspath{{images/}}
\usepackage[section]{placeins}

% Загрузите subcaption ДО настройки \thesubfigure
\usepackage{subcaption}

% Настройки подписей
\captionnamefont{\normalfont}
\captiontitlefont{\normalfont}
\captiondelim{ --- }
\renewcommand{\figurename}{Рисунок}
\renewcommand{\thesubfigure}{\asbuk{subfigure}}  % Теперь будет работать
\setkeys{Gin}{width=\textwidth}

% Математика и научные пакеты
\usepackage{amsmath}
\usepackage{siunitx}
\usepackage[version=4]{mhchem}

% Таблицы
\usepackage{longtable,ltcaption}
\usepackage{multirow,makecell}
\usepackage{booktabs}

% Код и листинги
\usepackage{listings}
\usepackage{xcolor}
\usepackage{fvextra}
\usepackage{minted}

% Списки
\usepackage{enumitem}
\renewcommand*{\labelitemi}{\normalfont{--}}
\makeatletter
    \AddEnumerateCounter{\asbuk}{\russian@alph}
\makeatother
\renewcommand{\labelenumii}{\asbuk{enumii})}
\renewcommand{\labelenumiii}{\arabic{enumiii})}
\setlist{noitemsep, leftmargin=*}
\setlist[1]{labelindent=\parindent}
\setlist[2]{leftmargin=\parindent}
\setlist[3]{leftmargin=\parindent}

% Библиография
\usepackage{csquotes}
\usepackage[
backend=biber,
bibencoding=utf8,
sorting=none,
style=gost-numeric,
language=auto,
autolang=other,
sortcites=true,
movenames=false,
maxnames=5,
minnames=3,
doi=false,
isbn=false,
]{biblatex}[2016/09/17]
\DeclareDelimFormat{bibinitdelim}{}
\addbibresource{bibl.bib}

% Дополнительные пакеты
\usepackage{framed}
\usepackage{pdfpages}
\usepackage{microtype}
\usepackage{tcolorbox}
\usepackage{hyphenat}
\usepackage{soul}

% Типографские настройки
\tolerance 1414
\hbadness 1414
\emergencystretch 1.5em
\hfuzz 0.3pt
\vfuzz \hfuzz
\clubpenalty=10000
\widowpenalty=10000
\brokenpenalty=4991
\sloppy

% Русская нумерация
\makeatletter
    \def\russian@Alph#1{\ifcase#1\or
       А\or Б\or В\or Г\or Д\or Е\or Ж\or
       И\or К\or Л\or М\or Н\or
       П\or Р\or С\or Т\or У\or Ф\or Х\or
       Ц\or Ш\or Щ\or Э\or Ю\or Я\else\xpg@ill@value{#1}{russian@Alph}\fi}
    \def\russian@alph#1{\ifcase#1\or
       а\or б\or в\or г\or д\or е\or ж\or
       и\or к\or л\or м\or н\or
       п\or р\or с\or т\or У\or Ф\or Х\or
       ц\or ш\or щ\or э\or ю\or я\else\xpg@ill@value{#1}{russian@alph}\fi}
\makeatother

% Счетчики
\counterwithout{figure}{chapter}
\counterwithout{equation}{chapter}
\counterwithout{table}{chapter}

% Гиперссылки
\usepackage{hyperref}
\hypersetup{
    colorlinks=true,
    linktoc=all,
    linktocpage=true,
    linkcolor=red,
    citecolor=red
}
\usepackage{bookmark}