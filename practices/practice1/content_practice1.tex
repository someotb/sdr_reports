\chapter[Практика: Архитектура SDR системы]{Практика\\ Архитектура SDR системы\\ Установка ПО, Настройка устройства}

\begin{leftbar}
\href{https://github.com/someotb/sdr_reports/tree/main/practices/practice1}{Ссылка на GitHub}
\end{leftbar}

\section*{Цель практики:}
Узнать, что такое SDR, изучить принципы его работы и внутреннюю
архитектуру на базовом уровне. Познакомиться с инструментом GNU Radio и
создать с его помощью программу для SDR, позволяющую принимать радио.

\section*{Краткие теоретические сведения}

Что такое SDR?

\begin{figure}[H]
    \centering
    \includegraphics[width=0.35\textwidth]{ADALM-Pluto_medium_block_diagram.png}
    \caption{ADALM Pluto}
\end{figure}

\begin{figure}[H]
    \centering
    \includegraphics[width=0.7\textwidth]{adalm_pluto.png}
    \caption{ADALM Pluto}
\end{figure}

Обучающая платформа PlutoSDR может взаимодействовать с:

\begin{enumerate}
    \item Matlab, Simulink
    \item GNU Radio
    \item C, C++ при помощи дополнительных библиотек
    \item C\#
    \item Среда языка Python
\end{enumerate}

В начале данного курса мы наладим взаимодействие PlutoSDR с языком Python. Далее напишем программы под другие платформы, сравним разницу по времени обработки сигналов, простоте написания кода и редактирования.

\section*{Чип AD9363}

Программируемый РЧ приемопередатчик, возможности которого позволяют использовать его для построения микро- (фемто-) сот мобильной связи 3G, 4G и 5G (в некоторых конфигурациях).

\begin{figure}[H]
    \centering
    \includegraphics[width=0.9\textwidth]{xilinx_zynq.png}
    \caption{ПЛИС Xilinx Zynq}
\end{figure}

\subsection*{Основные характеристики}

\begin{itemize}
    \item \textbf{12-битный} ЦАП/АЦП (Цифро-аналоговый/Аналого-цифровой Преобразователь)
    \item Поддерживаемые несущие частоты от 90 [МГц] до \textbf{3.8} [ГГц]
    \item Поддерживает временной и частотный дуплексы (\textbf{TDD}, \textbf{FDD})
    \item Ширина полосы частот: \textbf{20} [МГц]
    \item Шумы в приемнике: \textbf{3} [dB]
    \item EVM (Error Vector Magnitude): \textbf{-34} [dB]
    \item Tx noise: \textbf{< -157} [dBm/Hz]
    \item 2 \textbf{Rx}, 2 \textbf{Tx}
\end{itemize}

\section*{Структурная схема Zynq}

Каждый Zynq состоит из одного или двух ядер ARM Cortex-A9 (ARM v7), кэш L1 у каждого ядра свой, кэш L2 общий. Поддерживаемая оперативная память имеет стандарты DDR3, DDR3L, DDR2, LPDDR-2. Максимальный объем оперативной памяти равен 1 Гбайт (2 микросхемы по 4 Гбит). Максимальная тактовая частота оперативной памяти 525 МГц. Операционные системы: Standalone (bare-metal) и Petalinux. Процессорный модуль общается с внешним миром и программируемой логикой с помощью портов, объединенных в группы:

\begin{itemize}
    \item MIO (multiplexed I/O)
    \item EMIO (extended multiplexed I/O)
    \item GP (General-Purpose Ports)
    \item HP (High-Performance Ports)
    \item ACP (Accelerator coherency port)
\end{itemize}

\begin{figure}[H]
    \centering
    \includegraphics[width=0.9\textwidth]{zynq_structer.png}
    \caption{Структурная схема Zynq}
\end{figure}

\section*{Схема интерфейсов Zynq}

\textit{*Буквы S и M у порта обозначают соответственно Slave и Master.}

Так как, в одном корпусе Zynq реализованы и процессорный модуль и программируемая логика, есть выводы, которые относятся к процессорному модулю и выводы, которые относятся к программируемой логике.

\section*{Порты}

Порты MIO представляют собой многофункциональные порты ввода-вывода, непосредственно подключенные к выводам процессорной системы (Processing System, PS). Ключевые характеристики:

\begin{itemize}
    \item \textbf{Количество:} 54 порта (в большинстве конфигураций)
    \item \textbf{Назначение:} подключение периферийных устройств PS к внешним выводам кристалла
    \item \textbf{Особенности:} мультиплексирование функций на одних и тех же физических выводах
\end{itemize}

\subsection*{MIO}

Порты MIO подключены к выводам процессора. С помощью MIO могут быть подключены следующие периферийные устройства процессорного модуля:

\begin{itemize}
    \item USB-контроллер – 2 шт
    \item Gigabit Ethernet контроллер – 2 шт
    \item SD/SDIO контроллер – 2 шт
    \item UART – 2 шт
    \item CAN – 2 шт
    \item I2C – 2 шт
    \item SPI – 2 шт
    \item GPIO. Все выводы можно использовать как выводы общего назначения
\end{itemize}

Так же, к MIO могут быть подключены следующие устройства памяти процессорного модуля:

\begin{itemize}
    \item QSPI контроллер
    \item ONFI контроллер
    \item SRAM/NOR контроллер
\end{itemize}

Количество MIO портов равно 54 (за исключением некоторых микросхем в корпусе CLG225, там еще меньше). Поэтому все сразу включить не удастся. Для решения этой проблемы существует группа портов EMIO.

\section*{Выполнение}

\section*{Вывод}
Было изучено создание сигналов и работа с библиотеками Python для SDR.