\chapter[Практика: Архитектура SDR системы]{Практика\\ Архитектура SDR системы\\ Установка ПО, Настройка устройства}

\begin{leftbar}
\href{https://github.com/someotb/sdr_reports/tree/main/practices/practice1}{Ссылка на GitHub}
\end{leftbar}

\section*{Цель практики:}
Узнать, что такое SDR, изучить принципы его работы и внутреннюю
архитектуру на базовом уровне. Познакомиться с инструментом GNU Radio и
создать с его помощью программу для SDR, позволяющую принимать радио.

\section*{Краткие теоретические сведения}

Что такое SDR?

\begin{figure}[H]
    \centering
    \includegraphics[width=0.35\textwidth]{ADALM-Pluto_medium_block_diagram.png}
    \caption{ADALM Pluto}
\end{figure}

\begin{figure}[H]
    \centering
    \includegraphics[width=0.7\textwidth]{adalm_pluto.png}
    \caption{ADALM Pluto}
\end{figure}

Обучающая платформа PlutoSDR может взаимодействовать с:

\begin{enumerate}
    \item Matlab, Simulink
    \item GNU Radio
    \item C, C++ при помощи дополнительных библиотек
    \item C\#
    \item Среда языка Python
\end{enumerate}

В начале данного курса мы наладим взаимодействие PlutoSDR с языком Python. Далее напишем программы под другие платформы, сравним разницу по времени обработки сигналов, простоте написания кода и редактирования.

\section*{Чип AD9363}

Программируемый РЧ приемопередатчик, возможности которого позволяют использовать его для построения микро- (фемто-) сот мобильной связи 3G, 4G и 5G (в некоторых конфигурациях).

\begin{figure}[H]
    \centering
    \includegraphics[width=0.9\textwidth]{xilinx_zynq.png}
    \caption{ПЛИС Xilinx Zynq}
\end{figure}

\subsection*{Основные характеристики}

\begin{itemize}
    \item \textbf{12-битный} ЦАП/АЦП (Цифро-аналоговый/Аналого-цифровой Преобразователь)
    \item Поддерживаемые несущие частоты от 90 [МГц] до \textbf{3.8} [ГГц]
    \item Поддерживает временной и частотный дуплексы (\textbf{TDD}, \textbf{FDD})
    \item Ширина полосы частот: \textbf{20} [МГц]
    \item Шумы в приемнике: \textbf{3} [dB]
    \item EVM (Error Vector Magnitude): \textbf{-34} [dB]
    \item Tx noise: \textbf{< -157} [dBm/Hz]
    \item 2 \textbf{Rx}, 2 \textbf{Tx}
\end{itemize}

\section*{Структурная схема Zynq}

Каждый Zynq состоит из одного или двух ядер ARM Cortex-A9 (ARM v7), кэш L1 у каждого ядра свой, кэш L2 общий. Поддерживаемая оперативная память имеет стандарты DDR3, DDR3L, DDR2, LPDDR-2. Максимальный объем оперативной памяти равен 1 Гбайт (2 микросхемы по 4 Гбит). Максимальная тактовая частота оперативной памяти 525 МГц. Операционные системы: Standalone (bare-metal) и Petalinux. Процессорный модуль общается с внешним миром и программируемой логикой с помощью портов, объединенных в группы:

\begin{itemize}
    \item MIO (multiplexed I/O)
    \item EMIO (extended multiplexed I/O)
    \item GP (General-Purpose Ports)
    \item HP (High-Performance Ports)
    \item ACP (Accelerator coherency port)
\end{itemize}

\begin{figure}[H]
    \centering
    \includegraphics[width=0.9\textwidth]{zynq_structer.png}
    \caption{Структурная схема Zynq}
\end{figure}

\section*{Схема интерфейсов Zynq}

\textit{*Буквы S и M у порта обозначают соответственно Slave и Master.}

Так как, в одном корпусе Zynq реализованы и процессорный модуль и программируемая логика, есть выводы, которые относятся к процессорному модулю и выводы, которые относятся к программируемой логике.

\section*{Порты}

Порты MIO представляют собой многофункциональные порты ввода-вывода, непосредственно подключенные к выводам процессорной системы (Processing System, PS). Ключевые характеристики:

\begin{itemize}
    \item \textbf{Количество:} 54 порта (в большинстве конфигураций)
    \item \textbf{Назначение:} подключение периферийных устройств PS к внешним выводам кристалла
    \item \textbf{Особенности:} мультиплексирование функций на одних и тех же физических выводах
\end{itemize}

\subsection*{MIO}

Порты MIO подключены к выводам процессора. С помощью MIO могут быть подключены следующие периферийные устройства процессорного модуля:

\begin{itemize}
    \item USB-контроллер – 2 шт
    \item Gigabit Ethernet контроллер – 2 шт
    \item SD/SDIO контроллер – 2 шт
    \item UART – 2 шт
    \item CAN – 2 шт
    \item I2C – 2 шт
    \item SPI – 2 шт
    \item GPIO. Все выводы можно использовать как выводы общего назначения
\end{itemize}

Так же, к MIO могут быть подключены следующие устройства памяти процессорного модуля:

\begin{itemize}
    \item QSPI контроллер
    \item ONFI контроллер
    \item SRAM/NOR контроллер
\end{itemize}

Количество MIO портов равно 54 (за исключением некоторых микросхем в корпусе CLG225, там еще меньше). Поэтому все сразу включить не удастся. Для решения этой проблемы существует группа портов EMIO.

\section*{Выполнение}

Практическая работы выполнялась в программе GNU Radio. GNU Radio - это открытая платформа для разработки программных решений в области SDR.
Её ключевое преимущество — визуальное проектирование радиоэлектронных систем с помощью готовых блоков, что избавляет от необходимости
программировать. Собранные проекты работают непосредственно с SDR-оборудованием, таким как Adalm-Pluto или LimeSDR.

Библиотека GNU Radio содержит обширный набор компонентов для цифровой обработки сигналов (ЦОС). Сами модули разработаны на C++
для обеспечения высокой производительности, а их соединение и управление проектом осуществляется с помощью Python.
Разработку можно вести как программно, используя API, так и визуально — в графической среде GNU Radio Companion (GRC).

\subsection*{Создание схемы в GNU Radio}

\subsubsection*{Блок Options (Параметры проекта)}
Данный блок определяет основные настройки всего проекта. Наиболее важные параметры:

\begin{itemize}
    \item \textbf{Output Language (Язык выходного кода)}: определяет язык генерации исполняемой программы (Python или C++)
    \item \textbf{Generate Options (Опции генерации)}: задаёт тип используемого интерфейса
\end{itemize}

\subsubsection*{Блок Variable (Переменная)}
Служит для объявления переменных. Переменная имеет имя (ID) и значение. В проекте часто определяется:

\begin{verbatim}
Variable ID: samp_rate
Value: 2.4M
\end{verbatim}

\subsubsection*{Блок QT GUI Range (Ползунок)}
Создает регулируемый ползунок для динамического изменения параметров во время работы программы.

\begin{itemize}
    \item \textbf{Default Value}: значение при запуске
    \item \textbf{Start/Stop}: диапазон значений
    \item \textbf{Step}: шаг изменения
\end{itemize}

\subsubsection*{Блок PlutoSDR Source (Источник)}
Обеспечивает подключение к SDR-модулю ADALM-Pluto и управление его параметрами.

Основные параметры:
\begin{itemize}
    \item \textbf{IIO context URI}: IP-адрес устройства
    \item \textbf{Sample Rate}: частота дискретизации АЦП
    \item \textbf{LO Frequency}: частота гетеродина
    \item \textbf{Buffer size}: размер буфера
\end{itemize}

\subsubsection*{Блок Low Pass Filter (ФНЧ)}
Фильтр нижних частот для ограничения полосы пропускания сигнала.

Параметры:
\begin{itemize}
    \item \textbf{Decimation}: коэффициент уменьшения частоты дискретизации
    \item \textbf{Cutoff Freq}: частота среза
    \item \textbf{Sample Rate}: исходная частота дискретизации
\end{itemize}

\subsubsection*{Блок QT GUI Frequency Sink (Спектр)}
Обеспечивает визуализацию спектра сигнала в реальном времени.

\subsubsection*{Блок QT GUI Time Sink (Осциллограф)}
Отображает сигнал во временной области.

\subsubsection*{Блок WBFM Receive (FM-демодулятор)}
Выполняет демодуляцию широкополосного FM-сигнала.

\subsubsection*{Блок Audio Sink (Выход на звуковую карту)}
Направляет аудиопоток на звуковую карту компьютера.

\subsection*{Сборка системы}
Все блоки соединяются в графическом редакторе GRC, формируя законченную схему FM-приёмника.
Программа позволяет в реальном времени:

\begin{itemize}
    \item Принимать радиопередачи
    \item Наблюдать спектр и форму сигнала
    \item Регулировать частоту настройки
\end{itemize}

\begin{figure}[H]
    \centering
    \includegraphics[width=0.9\textwidth]{gnumodel.png}
    \caption{Собранная модель}
\end{figure}

Код собиратся автоматически исходя из модели собранной из блоков в интерфейсе GNU Radio. И с помощью данной модели получилось поймать
сигнал одной радио станции, которую можно было слушать, на других частотах звук был сильно искажен.
На радио станции пока слушал, понравилась одна песня: \href{https://www.youtube.com/watch?v=HWxCyJPMicU}{Девочка в платьице белом - Мюзикола}.

\begin{figure}[H]
    \centering
    \includegraphics[width=0.9\textwidth]{resgnuradio.png}
    \caption{Результаты}
\end{figure}

\section*{Вывод}
В результате работы были изучены принципы SDR, исследована архитектура ADALM Pluto и освоен инструмент GNU Radio.
Разработана программа для приёма и воспроизведения передач в FM-диапазоне.
